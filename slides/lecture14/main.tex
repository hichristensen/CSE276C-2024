\documentclass[10pt]{beamer}

\mode<presentation>
{
  \usetheme[height=1.25cm]{Madrid}
  \setbeamertemplate{navigation symbols}{}
  \setbeamercolor{alerted text}{fg=illini}
}

\graphicspath{{figs/}}

\usebackgroundtemplate{\includegraphics[width=\paperwidth,height=\paperheight]{uc-background}}

\usepackage[english]{babel}
\usepackage{epsfig,subfigure,bm}
\usepackage{multimedia}
\usepackage{psfrag}
\usepackage{animate}

% \usefonttheme{metropolis} % default family is serif
%%%%%% Begin of my macros and options

\setbeamertemplate{section in toc shaded}[default][55]
\setbeamertemplate{subsection in toc shaded}[default][55]
\setbeamercolor{block title}{fg=white,bg=illini}
\setbeamercolor{block body}{fg=black,bg=mygrey}

\setbeamercolor{emphprimary}{fg=CBlue}
\setbeamercolor{emphsecondary}{fg=illini}
\setbeamercolor{emphtertiary}{fg=mygreen}
\definecolor{darkForestGreen}{rgb}{.1,1,.1}
\definecolor{veryLightGray}{rgb}{.9,.9,.9}
\definecolor{greenApple}{rgb}{.3,.9,.3}

\setbeamercolor{title}{bg=CBlue}

\usepackage{amsmath,amssymb,amsxtra,amsthm}
\usepackage{algorithm,algorithmic}
\usepackage{natbib}
\usepackage{bibentry}
\usepackage{xspace}
\usepackage{changepage}

\definecolor{myblue}{rgb}{.2,.2,.7}
\definecolor{myred}{rgb}{.7,.2,.2}
\definecolor{mygreen}{rgb}{.2,.7,.2}
\definecolor{mygrey}{rgb}{0.9,0.9,0.9}
\definecolor{CBlue}{cmyk}{1,0.25,0,0}
\definecolor{illini}{rgb}{0.98,0.4,0.05}
\definecolor{black}{cmyk}{0,0,0,1}

\newcommand{\myemph}[1]{{\usebeamercolor[fg]{emphprimary}
    \textbf{#1}}}
\newcommand{\myemphalt}[1]{{\usebeamercolor[fg]{emphsecondary}
    \textbf{#1}}}

\graphicspath{{figs/}}

\title[Math for Robotics] % (optional, use only with long paper titles)
{CSE276C - Differential Geometry}

\author[H.~I. Christensen] % (optional, use only with lots of authors)
{Henrik I.~Christensen}
% - Give the names in the same order as the appear in the paper.  -
% Use the \inst{?} command only if the authors have different
% affiliation.

\AtBeginSection[]
{
   \begin{frame}
       \frametitle{Outline}
       \tableofcontents[currentsection]
   \end{frame}
}

\institute[UCSD] % (optional, but mostly needed)
{
  \begin{minipage}[c]{.2\textwidth}
    \includegraphics[width=.65\linewidth]{ucsealnew}%
  \end{minipage}%
  \begin{minipage}[c]{.6\textwidth}
    \small
%%    \begin{center}
      Computer Science and Engineering\\
      University of California, San Diego\\
%%    \end{center}

  \end{minipage}
%%  \vspace*{1ex}
}
%% - Use the \inst command only if there are several affiliations.
%% - Keep it simple, no one is interested in your street address.

\bigskip

\date[Nov 2024]% (optional, should be abbreviation of conference name)
{\small%
  November 2024}


\begin{document}

\nobibliography{/Users/hic/Dropbox/bibliography/bib-file}
\bibliographystyle{plain}

\begin{frame}[plain]
  \titlepage 
\end{frame}

\begin{frame}
  \frametitle{Introduction}
  \begin{itemize}
  \item We can only touch on the basics, but valuable to have basic
    knowledge
  \item Differential Geometry is all about moving on a curve /
    manifold
  \item Robotics is all about moving considering not only kinematics,
    but also dynamics
  \item What motion is possible in a particular space
  \end{itemize}
\end{frame}

\begin{frame}
  \frametitle{Basic Concepts}
  \begin{columns}
    \column{7cm}
    \begin{itemize}
    \item Tangent vector
      \begin{itemize}
      \item A vector anchored at a point p
      \item Set of possible vectors for p is termed tangent space $T_p$
      \end{itemize}
    \end{itemize}
    \column{4cm}
    \vfill
    \centerline{\includegraphics[width=3.8cm]{curve_binormal}}
    \vfill
  \end{columns}
\end{frame}

\begin{frame}
  \frametitle{Basic Concepts}
  \begin{columns}
    \column{7cm}
    \begin{itemize}
    \item Tangent Bundle
      \begin{itemize}
      \item A space along with its tangent vectors
      \item If $\mathbb{R}^n$ the underlying space and we have a
        tangent space of $\mathbb{R}^n$ anchored at each of the
        relevant points
      \item Space is then $\mathbb{R}^n \times \mathbb{R}^n$
      \item So a tangent bundle for a circle would be $S^1 \times \mathbb{R}^1$
      \end{itemize}
    \end{itemize}
    \column{4cm}
    \vfill
    \centerline{\includegraphics[width=3.8cm]{circle-bundle}}
    \vfill
  \end{columns}
\end{frame}

\begin{frame}
  \frametitle{Basic Concepts}
  \begin{itemize}
  \item Vector Field
    \begin{itemize}
    \item A function that maps a manifold to a tangent space
    \item $M \rightarrow T(M)$ and within it $p \rightarrow v_p \in T_p$ 
    \item Frequently denoted $V(p)$ or $V_p$ 
    \item A classic question: does a manifold has a continuously
      changing vector field that is non-zero?  \pause
    \item The circle example with $M=S^1$ is one such vector field
    \end{itemize}
  \end{itemize}
\end{frame}


\begin{frame}
  \frametitle{Geometry of curves in $\mathbb{R}^3$}
  \begin{itemize}
  \item Consider parameterized curves $\alpha (t) = (x(t), y(t), z(t))$
  \item In general a curve $\alpha$ is a mapping $\alpha: I \rightarrow \mathbb{R}^3$
  \item I is an interval in $\mathbb{R}$ sometimes we will write it as $(\alpha_1 (t), \alpha_2 (t), \alpha_3 (t))$
  \item In general $(x(t), y(t), z(t))$ are differentiable
  \item I.e., has derivatives of all orders throughout I
  \end{itemize}
\end{frame}

\begin{frame}
  \frametitle{A simple 2D example}
  \begin{columns}
    \column{4cm}
    \vfill
    \centerline{\includegraphics[width=3.8cm]{circle}}
    \vfill
    \column{7cm}
    \begin{itemize}
    \item $\alpha_1 (\theta) = (r \cos (\theta), r \sin (\theta))$
    \item $\theta \in [0, 2 \pi] = I$ OR
    \item $\alpha_2 (\theta) = (r \cos (2 \theta), r \sin (2\theta))$
    \item $\theta \in [0, \pi] = I$ 
    \end{itemize}
  \end{columns}
  \centering
  Different curves / parameterizations can have the same trace
\end{frame}

\begin{frame}
  \frametitle{Simple 3D curve}
  \begin{columns}
    \column{7cm}
    \begin{itemize}
    \item $\alpha (t) = ( a \cos(t), a \sin(t), b t)$, with $ t \in \mathbb{R}$
    \end{itemize}
    \column{4cm}
    \centerline{\includegraphics[width=3.8cm]{helix1}}
  \end{columns}
\end{frame}

\begin{frame}
  \frametitle{Velocity vector \& Arclength}
  \begin{itemize}
  \item The velocity vector of $\alpha$ at time t is the tangent vector of $\mathbb{R}^3$ given by
    \[ \alpha'(t) = ( \alpha'_1(t), \alpha'_2(t), \alpha'_3(t)) \]
  \item This vector is obviously also the tangent
  \item The speed of $\alpha$ is $v(t) = ||\alpha'(t)||$
  \item The arclength traversed between $t_0$ and $t_1$ is
    \[
      \int_{t_0}^{t_1} v(t) dt
    \]
  \item You can re-parameterize $\alpha (t)$ as $\beta (s)$ where s is
    the arc-length, which is the same as representing $\alpha$ at unit
    speed
  \end{itemize}
\end{frame}

\begin{frame}
  \frametitle{Simple Example --  Helix}
  \begin{itemize}
  \item Consider the helix: $\alpha(t) = ( r \cos(t), r \sin(t), q t )$ then
    \begin{itemize}
    \item Velocity: $\alpha'(t) = ( - r \sin (t), r \cos (t), q)$
    \item Speed: $ v(t) = \sqrt{ r^2 + q^2 } = c$ a constant
    \item Arc-length: $s(t) = \int_0^t c dt = ct$. Thus $t(s) = \frac{s}{c}$
    \item Re-parameterized: $\beta(s) = \alpha(\frac{s}{c}) = (r \cos (\frac{s}{c}), r \sin(\frac{s}{c}), q \frac{s}{c})$
    \end{itemize}
  \end{itemize}
\end{frame}

\begin{frame}
  \frametitle{Arclength?}
  \begin{itemize}
  \item So does the integral
    \[
      s(t) = \int_{t_0}^{t_1} ||\alpha'(t)|| dt
    \]
    always converge? \pause
  \item Some curves have infinite arclength (ex fractals - Koch Snowflake) 
    \centerline{\includegraphics[height=4cm]{koch-flake}}
  \end{itemize}
\end{frame}

\begin{frame}
  \frametitle{Vector fields of $\beta$}
  \begin{itemize}
  \item We can define a set of vector fields for $\beta$
    \begin{itemize}
    \item $T = \beta'$ the unit tangent field
    \item $N = \frac{T'}{||T'||}$ the principal normal vector field
    \item $B = T \times N$ called the bi-normal vector field of $\beta$
    \end{itemize}
  \item The quantity $||T'||$ is also named the curvature function $K(s) = ||T'(s)||$
  \item The triple (T,N,B) is called the Frenet Frame field of $\beta$
  \end{itemize}
\end{frame}

\begin{frame}
  \frametitle{Curvature}
  \begin{itemize}
  \item Let $\alpha: I \rightarrow \mathbb{R}^3$ be a curve parameterized by arclength
  \item Curvature is then defined as $||\alpha''(s)|| = K(s)$
  \item $\alpha'(s)$ -- the tangent vector of s
  \item $\alpha''(s)$ -- the change in the tangent vector 
  \item $R(s) = 1/K(s)$ -- is called the radius of curvature
  \end{itemize}
\end{frame}

\begin{frame}
  \frametitle{Simple examples}
  \begin{itemize}
  \item Straight line
    \[
      \begin{array}{rcl}
        \alpha(s)   & = & us + v, \mbox{~~~} u,v \in \mathbb{R}^2\\
        \alpha'(s)  & = & u\\
        \alpha''(s) & = & 0 \Rightarrow ||\alpha''(s)|| = 0\\
      \end{array}
    \]
  \item Circle
    \[
      \begin{array}{rcl}
        \alpha(s)   & = & (a \cos (s/a), a \sin (s/a)), \mbox{~~~} s \in [0,2 \pi a]\\
        \alpha'(s)  & = & ( -\sin(s/a), \cos (s/a))\\
        \alpha''(s) & = & (-\cos (s/a) / a, -\sin(s/a) / a) \Rightarrow ||\alpha''(s)|| = 1/a\\
      \end{array}
    \]
  \end{itemize}
\end{frame}

\begin{frame}
  \frametitle{Curvature examples}
  \begin{columns}
    \column{7cm}
    \begin{itemize}
    \item Cornu Spiral - K(s) = s
    \item Generalized Cornu Spirals - K(s) - Polynomial of s
    \end{itemize}
    \column{4cm}
    \centerline{\includegraphics[height=3cm]{CornuSpiral}}
    \centerline{\includegraphics[height=3cm]{CornuPolynomialSpirals}}
  \end{columns}
\end{frame}

\begin{frame}
  \frametitle{Normals}
  \begin{itemize}
  \item When $\alpha$ is parameterized by arc length
    \[
      \alpha'(s) \cdot \alpha'(s) = 1
    \]
  \item From Vector Calculus
    \begin{itemize}
    \item If f, g: $I \rightarrow \mathbb{R}^3$ and $f(t) \cdot  g(t) = const$ for all t
    \item then
      \[ f'(t) \cdot g(t) = -f(t) \cdot g'(t) \] for f * f this is only true for f'(t) f(t) = 0
    \end{itemize}
  \item This implies that
    \[ \alpha''(s) \cdot  \alpha'(s) = 0 \]
    or $\alpha''(s)$ is orthogonal to $\alpha'(s)$
  \item Its proportional to the normal of the curve
  \end{itemize}
\end{frame}

\begin{frame}
  \frametitle{Normals}
  \begin{columns}
    \column{7cm}
    \begin{itemize}
    \item $\alpha'(s) = T(s)$ -- Tangent Vector
    \item $||\alpha'(s)||$ -- arc length
    \item $\alpha''(s) = T'(s)$ -- normal direction
    \item $||\alpha''(s)||$ -- curvature
    \item If $||\alpha''(s) \neq 0$ then $\alpha''(s) = T'(s) = K(s) N(s)$
    \end{itemize}
    \column{4cm}
    \vfill
    \centerline{\includegraphics[width=3.8cm]{normals}}
    \vfill
  \end{columns}
\end{frame}

\begin{frame}
  \frametitle{Osculating Plane}
  \begin{columns}
    \column{4cm}
    \vfill
    \centerline{\includegraphics[width=3.5cm]{osculating-plane}}
    \vfill
    Source: M. Ben-Chen, Stanford
    \column{7cm}
    \begin{itemize}
    \item The local plane determined by the unit tangent and the
      normal vectors - T(s) and N(s) is call the osculating plane at s
    \end{itemize}
  \end{columns}
\end{frame}

\begin{frame}
  \frametitle{The Bi-normal Vector}
  \begin{columns}
    \column{7cm}
    \begin{itemize}
    \item The binormal is defined for $K(s) \neq 0$ by
      \[
        B(s) = T(s) \times N(s)
      \]
    \item The bi-normal defines the osculating plane
    \end{itemize}
    \column{4cm}
    \vfill
    \centerline{\includegraphics[width=3.5cm]{binormal}}
    \vfill
    Source: R. Gardner, ETSU
  \end{columns}
\end{frame}

\begin{frame}
  \frametitle{The Frenet Frame}
  \begin{columns}
    \column{5cm}
    \vfill
    \centerline{\includegraphics[width=3.8cm]{frenet-frame}}
    \vfill
    Source: A. J. Hanson, LBL
    \column{6cm}
    \begin{itemize}
    \item The system $\{ T(s), N(s), B(s) \}$ for an ortho-normal basis for $\mathbb{R}^3$ called the Fernet Frame
    \item The obvious question - How does it change along a curve? I.e., what are T'(s), N'(s), and B'(s)?
    \end{itemize}
  \end{columns}
\end{frame}

\begin{frame}
  \frametitle{T'(s)}
  \begin{itemize}
  \item We have already covered T'(s)
    \[
      T'(s) = K(s) N(s)
    \]

  \item As it is in the direction of N(s) it is orthogonal to B(s) and T(s). 
  \end{itemize}
\end{frame}

\begin{frame}
  \frametitle{N'(s)}
  \begin{itemize}
  \item We know that $N(s) \cdot N(s) = 1$ 
  \item From our earlier lemma (vector calculus) $N'(s) \cdot N(s) = 0$
  \item We know $N(s) \cdot T(s) = 0$ from the lemma $N'(s)\cdot T(s) = -N(s) \cdot T'(s)$
  \item Given $K(s) = N(s) \cdot T'(s)$
  \item It must be true that $N'(s) \cdot T(s) = -K(s)$
  \end{itemize}
\end{frame}

\begin{frame}
  \frametitle{Torsion}
  \begin{itemize}
  \item For the parameterized curve $\alpha: I \rightarrow \mathbb{R}^3$ the torsion of $\alpha$ is defined by
    \[
      \tau(s) = N'(s) \cdot B(s)
    \]
  \item We can then express
    \[
      N'(s) = K(s) T(s) + \tau (s) B(s)
    \]
  \end{itemize}
\end{frame}

\begin{frame}
  \frametitle{Curvature vs Torsion}
  \begin{itemize}
  \item \myemph{Curvature} indicates how much the normal changes in the direction of the tangent
  \item \myemph{Torsion} indicates how much the normal change in the direction orthogonal to the osculating plane
  \item Curvature is always positive, the torsion can be negative
  \item Neither depend on the choice of parameterization 
  \end{itemize}
\end{frame}

\begin{frame}
  \frametitle{B'(s)}
  \begin{itemize}
  \item We know that $B(s) \cdot B(s) = 1$
  \item From the lemma we know $B'(s) \cdot B(s) = 0$
  \item We further know: $B(s) \cdot T(s) = 0$ and $B(s) \cdot N(s) = 0$
  \item From the lemma:
    \[
      B'(s) \cdot T(s) = -B(s) \cdot T'(s) = B(s) \cdot K(s) N(s) = 0
    \]
  \item We get
    \[
      B'(s) \cdot N(s) = -B(s) \cdot N'(s) = - \tau(s)
    \] and from this we have
    \[
      B'(s) = - \tau(s) N(s)
    \]
  \end{itemize}
\end{frame}

\begin{frame}
  \frametitle{The Frenet Formulas}
  \[
    \begin{array}{rclll}
      T'(s)   &=& & K(s) N(s )& \\
      N'(s)   &=&-K(s) T(s) && + \tau(s) B(s)\\
      B'(s)   & =& & -\tau(s) N(s) \\
    \end{array}
  \]
  In Matrix Form
  \[
    \left(
      \begin{array}{ccc}
        | & | & | \\
        T'(s) & N'(s) & B'(s)\\
        | & | & | \\
      \end{array}
    \right) =
    \left(
      \begin{array}{ccc}
        | & | & | \\
        T(s) & N(s) & B(s)\\
        | & | & | \\
      \end{array}
    \right)
    \left(
      \begin{array}{ccc}
        0 & K(s)& 0 \\
        K(s) & 0 & -\tau(s)\\
        0  & \tau(s) & 0\\
      \end{array}
    \right)
  \]
\end{frame}

\begin{frame}
  \frametitle{Example - Back to the helix}
  \begin{itemize}
  \item For: $\alpha(t) = (a \cos(t), a \sin(t), bt)$
  \item Re-parameterized: $\alpha(s) = (a \cos(s/c), a \sin(s/c), b s/c)$ where $c = \sqrt{a^2 + b^2}$
  \item Curvature is then: $K(s) = \frac{a}{a^2+b^2}$
  \item Torsion is then $\tau(s) = \frac{}{a^2+b^2}$
  \item Note for this example both curvature and torsion are constants
  \end{itemize}
\end{frame}

\begin{frame}
  \frametitle{Covariant Derivatives and Lie Brackets}
  \begin{itemize}
  \item Suppose $V \& W$ are two vector fields in $\mathbb{R}^n$ so
    that for each point $p \in \mathbb{R}^n$ $V(p) \mbox{ and } W(p)$
    are vectors in $\mathbb{R}^n$
  \item The \myemph{covariant derivative} of W wrt V is
    \[
      (\nabla_v W)(p) = \frac{d}{dt} W(p + t V_p)|_{t = 0}
    \]
  \item $\nabla_v W$ measures the change in W as one moves along V
  \end{itemize}
\end{frame}

\begin{frame}
  \frametitle{Examples - covariant derivatives}
  \begin{itemize}
  \item In $\mathbb{R}^2$ W(p) = (1,0) and V(p) = (0,1) forall p
  \item The $\nabla_v W = \nabla_w V = 0$ 
  \item For a circle in 2D, $p = (x, y) \in \mathbb{R}^2$
    \[
      W = \frac{(x,y)}{\sqrt{x^2 + y^2}} \mbox{~ and ~}
      V = \frac{(-y, x)}{\sqrt{x^2+y^2}}
    \]
  \item Then $\nabla_v W = \frac{v}{\sqrt{x^2 + y^2}}$ and of course
    $\nabla_w V = 0$
  \end{itemize}
\end{frame}

\begin{frame}
  \frametitle{A few things about covariant derivatives}
  \begin{itemize}
  \item $\nabla_v W$ is an n-dimensional vector 
  \item $\nabla_v (a W + b U) = a \nabla_v W + b \nabla_v U$
  \item $\nabla_{fV+gU} W = f \nabla_v W + g \nabla_u W$
  \end{itemize}
\end{frame}

\begin{frame}
  \frametitle{Lie Bracket}
  \begin{itemize}
  \item The \myemph{Lie Bracket} [V, W] of the two vector fields is
    defined to be
    \[ [V, W] = \nabla_V W - \nabla_W V \]
  \item Basically measure flow in the directions of V, -V, W, -W
  \item Lets illustrate this with a real robot example 
  \end{itemize}
\end{frame}

\begin{frame}
  \frametitle{Parallel Parking}
  \begin{columns}
    \column{5cm}
    \vfill
    \centerline{\includegraphics[width=4.9cm]{parallel-parking}}
    \vfill
    \column{6cm}
    \begin{itemize}
    \item The configuration - $(x, y, \theta)$
    \item The controls are $(v, \phi)$
    \item The controls are
      \[
        \begin{array}{rcl}
          \dot{x}& = & v \cos \phi \cos \theta\\
          \dot{y}& = & v \cos \phi \sin \theta\\
          \dot{\theta} & = & \frac{v}{l} \sin \phi\\
        \end{array}
      \]
    \item We can consider nominal motion $(1, \phi_1)$ and $(1, \phi_2)$ as wheel directions
    \end{itemize}
  \end{columns}
\end{frame}
\begin{frame}
  \frametitle{Parallel Parking - Cont}
  \begin{itemize}
    \item Two vector fields
      \[ V_i = V_i(x,y,\theta) =
        (\cos \phi_i \cos\theta, \cos\phi_i \sin\theta, \frac{\sin \phi_i}{l})
      \]
    \item Then
      \[
        \nabla_{V_1} V_2 = (\nabla(\cos\phi_1 \cos \theta) V_2,
        \nabla(\cos \phi_1 \sin \theta) V_2, \nabla(\frac{\sin \phi_1}{l}) V2)
      \] skipping calculations
      \[
        \nabla_{V_1} V2 = \frac{\sin \phi_1 \cos \phi_2}{l}(-\sin \theta, \cos \theta, 0)
      \]  and similarly for the 
      \[
        [V_1, V_2] = \frac{\sin(\phi_1 - \phi_2)}{l} (-\sin \theta, \cos \theta, 0)
      \] So we can move perpendicular to the axis as long as $(\phi_1-\phi_2)\neq 0 $
  \end{itemize}
\end{frame}

\begin{frame}
  \frametitle{Moving to manifolds}
  \begin{itemize}
  \item Smooth Manifolds
    \begin{itemize}
    \item A manifold is a set M with an associated one-to-one map $\phi: U \rightarrow M$ from an open subset $U \subset \mathbb{R}^m$ called a global chart or coordinate system of M
    \end{itemize}
    \end{itemize}
      \centerline{ \includegraphics[width=3.5cm]{basic-manifold} \parbox[c][2cm][c]{2cm}{\vfill\centering $\rightarrow$\\ \vfill} \includegraphics[width=3.5cm]{gauss-mixture} }
\end{frame}

\begin{frame}
  \frametitle{Smooth Manifolds}
  \begin{itemize}
  \item A smooth manifold is a pair $(M,\mathcal{A})$ where:
    \begin{itemize}
    \item M is a set
    \item $\mathcal{A}$ is a family of 1-1 charts:
      $\phi: U \rightarrow M$ from some open subset
      $U = U_{\phi} \subset \mathbb{R}^m$ for M
    \end{itemize}
  \end{itemize}
\end{frame}

\begin{frame}
  \frametitle{Differentiable and smooth functions}
  \begin{itemize}
  \item $f : U \subset \mathbb{R}^n \rightarrow \mathbb{R}^q$
    \[
      (y_1, \ldots, y_q ) = f(x_1,\ldots,x_n)
    \]
  \item f is of a class $C^r$ if f has continuous partial derivatives
    \[
      \frac{\partial^{r_1+\ldots+r_n}y_k}{\partial x_1^{r_1} \ldots \partial x_n^{r_n}}
    \]
  \item If $r=\infty$, then f is \myemph{smooth}, the main focus in robotics
  \end{itemize}
\end{frame}


\begin{frame}
  \frametitle{Diffeomorphism}
  \begin{itemize}
  \item When n = q
    \begin{itemize}
    \item if f is 1-1, $f$ and $f^{-1}$ are both $C^r$
    \item $\Rightarrow f$ is a \myemph{$C^r$-diffeomorphism}
    \item Smooth diffemorphisms are simply referred as diffeomorphisms
    \end{itemize}
  \item Inverse Function Theorem:
    \begin{itemize}
    \item f diffeomorphism $\Rightarrow det(J_x f) \neq 0$
    \item $det(J_x f) \neq 0 \Rightarrow f$ is local diffeomorphism in a neighborhood of x
    \end{itemize}
  \end{itemize}
\end{frame}

\begin{frame}
  \frametitle{Example - Gaussian Distribution}
  \begin{itemize}
  \item The space of n-dimensional Gaussian distributions is a smooth manifold
  \item Global chart: $(\mu, \Sigma) \in \mathbb{R}^n \times \mathcal{P}(n)$
    \centerline{\includegraphics[height=3.5cm]{normal-distribution}}
  \end{itemize}
\end{frame}

\begin{frame}
  \frametitle{Manifolds can generate multiple charts}
  \begin{columns}
    \column{5cm}
    \includegraphics[width=4.5cm]{manifold_sphere}
    \column{7cm}
    \begin{itemize}
    \item The sphere $\mathcal{S}^2 = \{(x,y,x), x^2+y^2+z^2 = 1 \}$
      has multiple projections/charts
    \item We can project from the North Pole, of a point P = (x,y,z)
      given by
      \[ \phi(P) = \left( \frac{x}{1-z},\frac{y}{1-z} \right) \]
    \item is a large coordinate system around the south pole
    \end{itemize}
  \end{columns}
\end{frame}

\begin{frame}
  \frametitle{Manifolds requiring multiple chartss}
  \begin{columns}
    \column{6cm}
    \centerline{\Large The Moebius Strip}\\
    \centerline{\includegraphics[width=4cm]{moebius-strip}}
    \begin{center}
      \[ u \in [0, 2 \pi], v \in [-1/2, 1/2] \]
      \[
        \left(
          \begin{array}{c}
            \cos(u) \left( 1+\frac{1}{2} v \cos \left(\frac{u}{2}\right)\right)\\
            \sin(u) \left( 1+\frac{1}{2} v \cos \left(\frac{u}{2}\right)\right)\\
            \frac{1}{2} v \sin \left(\frac{u}{2} \right)\\ 
          \end{array}
        \right)
      \]
    \end{center}
    \column{6cm}
    \centerline{\Large 2D Torus}\\
    \centerline{\includegraphics[width=4cm]{torus}}
    \begin{center}
      \[ (u,v) \in [0,2 \pi]^2, R>> r > 0 \]
      \[ 
        \left(
          \begin{array}{c}
            \cos(u) \left( R + r \cos(v) \right)\\
            \sin(u) \left( R + r \cos(v) \right)\\
            r \sin(v)\\
          \end{array}
        \right)
      \]
    \end{center}
  \end{columns}
 \end{frame}

\begin{frame}
  \frametitle{Summary}
  \begin{itemize}
  \item Covering basics of movement along curves
  \item Many more derivations can be provided for movement on manifolds
  \item Covering basic characteristics of curves and manifolds
  \item Definition of the Frenet frame and associated characteristics
  \item Brief coverage of covariant derivatives and Lie bracket
  \end{itemize}
\end{frame}

\end{document}

%%% Local Variables:
%%% mode: latex
%%% TeX-master: t
%%% End:
